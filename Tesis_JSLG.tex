% Options for packages loaded elsewhere
\PassOptionsToPackage{unicode}{hyperref}
\PassOptionsToPackage{hyphens}{url}
\PassOptionsToPackage{dvipsnames,svgnames,x11names}{xcolor}
%
\documentclass[
  us-letterpaper,
]{scrreprt}

\usepackage{amsmath,amssymb}
\usepackage{iftex}
\ifPDFTeX
  \usepackage[T1]{fontenc}
  \usepackage[utf8]{inputenc}
  \usepackage{textcomp} % provide euro and other symbols
\else % if luatex or xetex
  \usepackage{unicode-math}
  \defaultfontfeatures{Scale=MatchLowercase}
  \defaultfontfeatures[\rmfamily]{Ligatures=TeX,Scale=1}
\fi
\usepackage{lmodern}
\ifPDFTeX\else  
    % xetex/luatex font selection
\fi
% Use upquote if available, for straight quotes in verbatim environments
\IfFileExists{upquote.sty}{\usepackage{upquote}}{}
\IfFileExists{microtype.sty}{% use microtype if available
  \usepackage[]{microtype}
  \UseMicrotypeSet[protrusion]{basicmath} % disable protrusion for tt fonts
}{}
\makeatletter
\@ifundefined{KOMAClassName}{% if non-KOMA class
  \IfFileExists{parskip.sty}{%
    \usepackage{parskip}
  }{% else
    \setlength{\parindent}{0pt}
    \setlength{\parskip}{6pt plus 2pt minus 1pt}}
}{% if KOMA class
  \KOMAoptions{parskip=half}}
\makeatother
\usepackage{xcolor}
\setlength{\emergencystretch}{3em} % prevent overfull lines
\setcounter{secnumdepth}{5}
% Make \paragraph and \subparagraph free-standing
\ifx\paragraph\undefined\else
  \let\oldparagraph\paragraph
  \renewcommand{\paragraph}[1]{\oldparagraph{#1}\mbox{}}
\fi
\ifx\subparagraph\undefined\else
  \let\oldsubparagraph\subparagraph
  \renewcommand{\subparagraph}[1]{\oldsubparagraph{#1}\mbox{}}
\fi


\providecommand{\tightlist}{%
  \setlength{\itemsep}{0pt}\setlength{\parskip}{0pt}}\usepackage{longtable,booktabs,array}
\usepackage{calc} % for calculating minipage widths
% Correct order of tables after \paragraph or \subparagraph
\usepackage{etoolbox}
\makeatletter
\patchcmd\longtable{\par}{\if@noskipsec\mbox{}\fi\par}{}{}
\makeatother
% Allow footnotes in longtable head/foot
\IfFileExists{footnotehyper.sty}{\usepackage{footnotehyper}}{\usepackage{footnote}}
\makesavenoteenv{longtable}
\usepackage{graphicx}
\makeatletter
\def\maxwidth{\ifdim\Gin@nat@width>\linewidth\linewidth\else\Gin@nat@width\fi}
\def\maxheight{\ifdim\Gin@nat@height>\textheight\textheight\else\Gin@nat@height\fi}
\makeatother
% Scale images if necessary, so that they will not overflow the page
% margins by default, and it is still possible to overwrite the defaults
% using explicit options in \includegraphics[width, height, ...]{}
\setkeys{Gin}{width=\maxwidth,height=\maxheight,keepaspectratio}
% Set default figure placement to htbp
\makeatletter
\def\fps@figure{htbp}
\makeatother

\usepackage{upgreek}
\usepackage{amsmath}
\usepackage{amssymb}
\newcommand{\dashedbox}[1]{
  \begin{tikzpicture}
    \node[draw, dashed, rounded corners=5pt, inner sep=10pt] {
      \begin{minipage}{0.8\textwidth} % Establece el ancho del minipage
        #1
      \end{minipage}
    };
  \end{tikzpicture}
}
\makeatletter
\@ifpackageloaded{bookmark}{}{\usepackage{bookmark}}
\makeatother
\makeatletter
\@ifpackageloaded{caption}{}{\usepackage{caption}}
\AtBeginDocument{%
\ifdefined\contentsname
  \renewcommand*\contentsname{Tabla de contenidos}
\else
  \newcommand\contentsname{Tabla de contenidos}
\fi
\ifdefined\listfigurename
  \renewcommand*\listfigurename{Listado de Figuras}
\else
  \newcommand\listfigurename{Listado de Figuras}
\fi
\ifdefined\listtablename
  \renewcommand*\listtablename{Listado de Tablas}
\else
  \newcommand\listtablename{Listado de Tablas}
\fi
\ifdefined\figurename
  \renewcommand*\figurename{Figura}
\else
  \newcommand\figurename{Figura}
\fi
\ifdefined\tablename
  \renewcommand*\tablename{Tabla}
\else
  \newcommand\tablename{Tabla}
\fi
}
\@ifpackageloaded{float}{}{\usepackage{float}}
\floatstyle{ruled}
\@ifundefined{c@chapter}{\newfloat{codelisting}{h}{lop}}{\newfloat{codelisting}{h}{lop}[chapter]}
\floatname{codelisting}{Listado}
\newcommand*\listoflistings{\listof{codelisting}{Listado de Listados}}
\makeatother
\makeatletter
\makeatother
\makeatletter
\@ifpackageloaded{caption}{}{\usepackage{caption}}
\@ifpackageloaded{subcaption}{}{\usepackage{subcaption}}
\makeatother
\ifLuaTeX
\usepackage[bidi=basic]{babel}
\else
\usepackage[bidi=default]{babel}
\fi
\babelprovide[main,import]{spanish}
% get rid of language-specific shorthands (see #6817):
\let\LanguageShortHands\languageshorthands
\def\languageshorthands#1{}
\ifLuaTeX
  \usepackage{selnolig}  % disable illegal ligatures
\fi
\usepackage{bookmark}

\IfFileExists{xurl.sty}{\usepackage{xurl}}{} % add URL line breaks if available
\urlstyle{same} % disable monospaced font for URLs
\hypersetup{
  pdftitle={Análisis comparativo del desempeño en métodos para el pronóstico de series temporales},
  pdfauthor={Jennifer Sherlyn López García; Yofre Hernán García Gómez},
  pdflang={es},
  colorlinks=true,
  linkcolor={blue},
  filecolor={Maroon},
  citecolor={Blue},
  urlcolor={Blue},
  pdfcreator={LaTeX via pandoc}}

\title{Análisis comparativo del desempeño en métodos para el pronóstico
de series temporales}
\author{Jennifer Sherlyn López García \and Yofre Hernán García Gómez}
\date{2024-02-14}

\begin{document}
\begin{titlepage}
\hspace{-1.7cm} %Este comando es para mandar a la izquierda ;)
%Aquí empiezan los tres minipages de antes
\begin{minipage}[t][0.03\textheight][c]{0.22\textwidth}
        \includegraphics[width=4.0cm]{LOGO50.png}
\end{minipage}\hspace{0.9cm}
\begin{minipage}[t][0.03\textheight][c]{0.69\textwidth}
\begin{center}
                \textsc{\huge Universidad Autónoma de Chiapas}\\[0.3cm]
                \hrule height 2.5pt
                \vspace{0.2cm}
                \hrule height1pt
                \vspace{0.3cm}
                \textsc{\Large Facultad de Ciencias en Física y Matemáticas}
\end{center}
\end{minipage}\hspace{0.2cm}
\begin{minipage}[t][0.03\textheight][c]{0.2\textwidth}
		\includegraphics[width=2.7cm]{logofcfm.png}
\end{minipage}\\
%%%%%%%%%%%%%%%%%%Aquí comienzan los otros dos minipages del título%%%%%%%%%%%%%%%%%%%%%%%%%%%%%%%%%%%%%%%%%%%%%%%%%%%%%%%%%%%%%%%%%%%%%%%%%%%%%%%%%%%%%%%%%%%%%%
\begin{minipage}[t][0.93\textheight][c]{0.06\textwidth}
\vspace{60pt}
    \begin{center}
        \vrule width1pt height18cm
        \vspace{5mm}
        \vrule width2.5pt height18cm
        \vspace{5mm}
        \vrule width1pt height18cm
   \end{center}
\end{minipage}\hspace{1.3cm} %Esto mueve las letras que tienes ahí
\begin{minipage}[t][0.95\textheight][c]{0.76\textwidth}

            \begin{center}
                {\Large\bfseries Análisis comparativo del desempeño en métodos para el pronóstico de series temporales}\\[2cm]
                \textsc{\huge \textbf{T\, E\, S\, I\, S}}\\[1.5cm]
                \textsc{\large QUE PARA OBTENER EL TÍTULO DE:}\\[0.3cm]
                \textbf{\textsc{LICENCIADA EN MATEMÁTICAS APLICADAS}}\\[1.5cm]
                \textsc{\large PRESENTA:}\\[0.3cm]
                \textbf{\textsc{\large {JENNIFER SHERLYN LÓPEZ GARCÍA}}}\\[2cm]
                {\large\scshape Director:\\[0.3cm]
                {\textbf{\large Dr. Yofre Hernán García Gómez }}}\\[2.0cm]
                \large{Tuxtla Gutiérrez, Chiapas a - de - del 2024.}

            \end{center}
\end{minipage}
\end{titlepage}

\pagebreak[2]

\chapter*{Dedicatoria}
\begin{flushright}
\textit{A mis padres, \\ por su inquebrantable fe en mí y por ser mi inspiración para perseguir la excelencia. Esta tesis es mi manera de agradecerles su amor, paciencia y sacrificio.}
\end{flushright}


\chapter*{Agradecimientos}
En primer lugar, deseo agradecer de todo corazón a mi director de tesis, el Dr. Yofre Hernán García Gómez. Gracias por compartir su vasto conocimiento conmigo durante estos años. El tiempo que ha invertido en mí ha sido el regalo más valioso que tengo, gracias a su guía y mentoría, he logrado adquirir conocimientos que nunca habría imaginado alcanzar por mi cuenta. Esta tesis es el fruto directo de su apoyo incondicional, y estoy profundamente agradecida por ello.

Asimismo, quiero expresar mi profundo agradecimiento a mis estimados profesores, especialmente a la Dra. Rosario y al Mtro. Elesban, cuyo conocimiento y dedicación han sido pilares en la realización de este trabajo. A la Dra. Rosario, le agradezco sinceramente por su disposición a escucharme, sus valiosos consejos y su constante motivación. El impacto de su enseñanza perdurará en mi carrera académica. Al Mtro. Elesban, le estoy infinitamente agradecido por su inquebrantable compromiso con mi aprendizaje. Su dedicación, incluso en las horas más tardías, ha sido un ejemplo de entrega y profesionalismo.

Además, mi gratitud se extiende a mis queridos amigos Gustavo, Mara, Jaziel, Jordi, Jairo, Darinel, Daniel, Luis, Christian, Christopher y Ángel. Su apoyo incondicional, sus abrazos reconfortantes y su paciencia han sido un sostén fundamental durante este proceso. Agradezco cada momento compartido, cada risa compartida y cada gesto de solidaridad. Ustedes son verdaderamente las mejores personas que uno pueda desear tener como amigos, y su contribución a este logro es inestimable. Especialmente, quiero reconocer a mi mejor amigo y compañero de vida, Kevin. Tu amor incondicional, tu comprensión infinita y tu paciencia sin límites durante los momentos de estrés y dedicación. Tu inspiración y motivación han sido un faro de luz en los días más oscuros, gracias por nunca dejarme caer. Gracias por celebrar cada logro a mi lado y por ser mi compañero fiel en este viaje. Esta tesis es también tuya, pues has sido parte integral de cada paso dado. Te amo.

Sin embargo, el más profundo agradecimiento lo reservo para mis padres, Yesenia e Hipólito. Su amor incondicional, su sacrificio y su constante apoyo han sido la fuerza motriz que me ha impulsado a alcanzar mis metas. Gracias por estar siempre ahí para mí, por brindarme todo lo que necesitaba y por ser modelos de responsabilidad y dedicación. No tengo palabras para expresar la profundidad de mi gratitud hacia ustedes; los amo con todo mi corazón.

A todos ustedes, mi más sincero agradecimiento. Este logro no habría sido posible sin su generosidad, apoyo y aliento constante.
\renewcommand*\contentsname{Tabla de contenidos}
{
\hypersetup{linkcolor=}
\setcounter{tocdepth}{2}
\tableofcontents
}
\bookmarksetup{startatroot}

\chapter*{Resumen}\label{resumen}
\addcontentsline{toc}{chapter}{Resumen}

\markboth{Resumen}{Resumen}

Esta tesis presenta un análisis comparativo de métodos de pronóstico
para datos de series temporales, centrándose en la aplicación del método
Holt-Winters y de las redes neuronales Perceptrón Multicapa (MLP). El
estudio abarca una revisión exhaustiva de la teoría del análisis de
series temporales y los fundamentos de las redes neuronales,
proporcionando una sólida base teórica para comprender las metodologías
empleadas.

La investigación examina el desempeño del método de Holt-Winters, una
técnica clásica de suavizado exponencial, y MLP, un enfoque potente de
modelado no lineal, en el pronóstico del número de casos confirmados y
fallecidos debido al COVID-19 en Irán. Al aplicar estos métodos a datos
del mundo real, la tesis evalúa su precisión, robustez y eficiencia
computacional en la captura de la compleja dinámica de la progresión de
la pandemia.

A través del análisis empírico y las evaluaciones comparativas, este
estudio tiene como objetivo proporcionar información sobre las
fortalezas y limitaciones de cada enfoque de pronóstico, contribuyendo
así al avance de las metodologías de pronóstico para datos de series
temporales, especialmente en el contexto de crisis de salud pública.

\bookmarksetup{startatroot}

\chapter*{Introducción}\label{introducciuxf3n}
\addcontentsline{toc}{chapter}{Introducción}

\markboth{Introducción}{Introducción}

La creciente disponibilidad de datos temporales en una amplia gama de
campos ha impulsado la necesidad de desarrollar métodos eficaces para
pronosticar su comportamiento futuro. En particular, el análisis y
pronóstico de series temporales desempeña un papel crucial en la toma de
decisiones en áreas como la economía, la meteorología, la ingeniería y
la salud pública. Con la aparición de la pandemia de COVID-19, la
capacidad de prever la evolución de la enfermedad se ha convertido en
una prioridad urgente para los responsables de la salud y los
planificadores de políticas.

Este trabajo se centra en el análisis comparativo de métodos de
pronóstico para series temporales, con un enfoque específico en el
método de Holt-Winters y las redes neuronales Perceptrón Multicapa
(MLP). Para comprender en profundidad estos enfoques, se presenta una
revisión exhaustiva de la teoría del análisis de series temporales y los
fundamentos de las redes neuronales, estableciendo así las bases
teóricas necesarias para su aplicación.

El método de Holt-Winters, basado en el suavizado exponencial, y MLP,
una técnica de modelado no lineal, son seleccionados como enfoques
principales debido a su amplia aplicación y capacidad para capturar
patrones complejos en los datos temporales. Se busca evaluar el
desempeño de estos métodos en la predicción del número de casos
confirmados y fallecidos por COVID-19 en Irán, utilizando datos reales
recopilados durante el curso de la pandemia.

El estudio se estructura en torno al desarrollo de modelos de pronóstico
basados en Holt-Winters y MLP, seguido de una comparación exhaustiva de
su precisión, robustez y eficiencia computacional. Además, se explora el
potencial de estas técnicas para proporcionar información valiosa en la
gestión de crisis de salud pública y la planificación de recursos.

En última instancia, se espera que este trabajo contribuya al avance de
las metodologías de pronóstico para datos de series temporales,
ofreciendo perspectivas significativas para la mejora de la capacidad
predictiva en situaciones críticas como la pandemia de COVID-19.



\end{document}
